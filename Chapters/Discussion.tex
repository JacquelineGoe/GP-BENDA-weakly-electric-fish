% --------------------------------------------------
% all species, all habitats
% --------------------------------------------------

Our goal was to gain more information about the habitats and the ecology of different species of weakly electric fish. We characterised habitats and day shelter of different free living weakly electric fish in a network of river channels in the LLanos of the Orinoco basin, Meta, Colombia. Most of the examined habitats contained stones. Roots were also found frequently, whereas habitats containing mud or plants were less frequent (fig.~\ref{fig:habitat_count}). Within all examined habitats, weakly electric fish of three different species  or genera were found: wave-type fish of the species \textit{Apteronotus macrostomus} and \textit{Eigenmannia virescens} and pulse-type fish. The different species strongly differed in their frequency of occurrence (fig.~\ref{fig:habitat_count_species}). 
The three observed species preferred different microhabitats with different habitat characteristics, such as ground texture, water flow and water depth. Additionally, in \textit{A. macrostomus} a relation between dominance and habitat selection was found. 

\subsection{Habitat characteristics and preferences}

In general, different species show different preferences for habitats and shelter \citep{redsalmon1995,sherry1989redstarts,downes1998heat}. This also holds true for weakly electric fish \citep{Hopkins_74,HAGEDORN1985,lissmann1965activity}. Our results show different (micro-) habitat preferences of different species of weakly electric fish.
During the day, pulsefish preferred plants and roots as shelter, but were seldom found between stones (fig.~\ref{fig:habitat_vs_flow}). Since the preferred habitat characteristics of pulsefish occurred less frequent compared with other characteristics (fig.~\ref{fig:habitat_count}), the over-all count of these animals was quite low (fig.~\ref{fig:habitat_vs_eod}). \textit{Eigenmannia} showed a preference for mainly roots, which corresponds with the literature \citep{Hopkins_74}. Even though quite many microhabitats containing roots were found (fig.~\ref{fig:habitat_count}), the over-all count of individuals of this species was quite low compared with \textit{Apteronotus} (fig.~\ref{fig:habitat_vs_eod}). \textit{A. macrostomus} inhabited mostly stacked stones, but was also found in roots, rarely in plants. Stony habitats were by far the most frequent. Accordingly, \textit{Apteronotus} was by far the most frequent species. In pure sand habitats no fish were found. Since weakly electric fish are nocturnal animals \citep{lissmann1965activity} that seek shelter during the day \citep{Hopkins_74}, it seems likely that microhabitats containing solely sand are poor hiding places for the three occurring species.

Water flow and water depth can also play a role in shelter or habitat selection \citep{aadland1993stream}. Pulsefish inhibited habitats with either very low water flow or no water flow at all (fig.~\ref{fig:habitat_vs_flow}).  Most of them were found in water depth between 20~and 40~cm (fig.~\ref{fig:habitat_vs_depth}). However, a clear preference for a certain water depth can not be seen, due to a small amount of recorded pulse-fish in diverse microhabitats.
\textit{Apteronotus} also showed no preference, neither for water depth nor water flow. This species was found in a large flow range from stagnant to fast moving water and a variety of different water depths.
\textit{Eigenmannia} was mainly found in slow moving water, but some individuals were also found in high flow areas. Regarding the water depth, \textit{Eigenmannia virescens} is the only species that seemed to prefer deeper water. However, this effect might be caused by measurement inaccuracies. Sometimes, especially if several animals were found within a microhabitat, it was impossible to tell at which depth the individuals appeared. For each recording and microhabitat, the denoted water depth corresponds only with the position of one individual (chpt.~\ref{sec:recordings}).

Over-all, \textit{Apteronotus} seems to be more opportunistic, occurring in a variety of microhabitats regardless of water depth and flow. Other species of weakly electric fish were rarely found in the preferred habitat of \textit{Apteronotus} (stones). This could indicate that next to same sex conspecifics \citep{raab2019}, \textit{Apteronotus} may also defend its shelter against other species. This potential inter-specific dominance behavior of \textbf{Apteronotus} might have a strong influence on the habitat selection of the occurring  species. Another factor in influencing the habitat selection of the different species might be the occurrence of food sources. It is likely that different species prefer different aquatic insect larvae \citep{marrero1991notas}, which are likely to occur in different frequencies in the observed microhabitats. To better understand the habitat preferences of the different species,  it is necessary to know why certain shelter are selected. Investigating inter-specific dominance and natural foraging behavior of the different species, would be a good next step.

% --------------------------------------------------
% only apteronoti
% --------------------------------------------------

\subsection{Dominance and habitat preferences in \textit{Apteronotus}}

\begin{itemize}
    \item meisten Tiere Stones, dann Roots (sowohl männchen als auch weibchen)
    \item grund: steine bevorzugtes habitat - evtl guter shelter oder großes futteraufkommen oder weil wir einfach doppelt so viele Stein habitate hatten als root. Würde das argument eher auf die dominanz beziehen, weil es eine Aussage ist die die Daten bestätigen
    
    \item männchen: dominantereste tiere ausschließlich in steinen
    \item grund: sie setzen sich auf grund der dominanz gegen weniger dominante tiere durch
    \item daraus resultierende gründe: weil sie dominant sind haben sie bessere shelter oder mehr futter was dann deren überlebenschancen erhöht und somit die fitness, was sich wiederum auf die dominanz ausüben könnte...
    \item (body size ~ food intake, body size ~ dominance in Apteronotus leptorhynchus (K.D. Dunlap and L.M. Oliveri Retreat site selection and social organization in captive electric fish, Apteronotus leptorhynchus)
    \item innerhalb der weibchen: signifikanter unterschied zwischen EODf der individuen stein-vs-wurzel
    \item grund: vllt der selbe wie bei männchen weil dominanz 
    
    \item mögliche hinweise in diesen quellen dafür:
    \item Retreat site selection and social organization in captive electric fish, Apteronotus leptorhynchus, K.D. Dunlap and L.M. Oliveri, Journal of Comparative Physiology A, 188 (2002), pp. 469-477
    \item Electrocommunication signals in female brown ghost electric knife fish, Apteronotus leptorhynchus, S.K. Tallarovic and H.H. Zakon, Journal of Comparative Physiology A, 188 (2002), pp. 649-657
    \item \textcolor{red}{Allgemein: es scheint dominanz sowohl in männchen als auch weibchen zu geben, nur in männchen eben stärker ausgeprägt...}
    \item ZUKÜNFTIGE EXPERIMENTE:
    \item gründe warum steine ein besseres habitat sind als wurzel sollte genauer erforscht werden
    \item dominanz verhalten und habitat selection weiter untersuchen, auch zwischen weibchen 
    \item die Untersuchungsergebnisse von till im freiland sich anzu schauen 
\end{itemize}{}

% ergebnisse - zusammenfassung
\textit{Apterontus macrostomus} is mainly found in stacked stones followed by roots, which can be partially observed under laboratory conditions \citep{raab2019}. We found that high frequency males exclusively were found in stony habitats. Also the frequency of females in stony habitat was on average higher then the ones in root habitats. High EOD frequency is characteristic for dominance in male \textit{A. macrostomus}. The higher the frequency the more dominant the male with correlates with size of the fish \citep{triefenbach_zakon2003}. 

% conclusion
This suggests that stony habitats are preferred hiding place over the day. Stony habitats seem to be save shelters or offers high food availability and play a important role in the wild. This indicates that stacked stone 
Dominant males stay longer in their preferred day time habitat. But it is also observed that individuals switch to a new shelter in regular intervals mostly over night as well as over the day \citep{raab2019}. 


To better understand this, it is necessary to investigate the long term behavior in the wild over day and night to see which shelters are chosen.


