% --------------------------------------------------
% all species, all habitats
% --------------------------------------------------

Our goal was to gain more information about the habitats and the ecology of different species of weakly electric fish. We characterised habitats and day shelter of different free living weakly electric fish in a network of river channels in the LLanos of the Orinoco basin, Meta, Colombia. Most habitats contained stones. Roots were also found frequently, whereas habitats with mud or plants were less frequent (fig.~\ref{fig:habitat_count}). Within the examined habitats, weakly electric fish of three species  or genera were found: wave-type fish of the species \textit{Apteronotus macrostomus} and \textit{Eigenmannia virescens} and pulse-type fish. The different species strongly differed in their frequency of occurrence (fig.~\ref{fig:habitat_count_species}). \textcolor{red}{(VGL LITERATUR). 
(Übergang?)}
The observed species preferred different micro-habitats with different habitat characteristics, such as ground texture, water flow and water depth. Additionally, in \textit{A. macrostomus} a relation between dominance and habitat selection was found. 

\subsection{Habitat characteristics and preferences}

Different species show selection towards different habitats and shelters \citep{redsalmon1995,sherry1989redstarts,downes1998heat}. This also holds true for weakly electric fish \citep{Hopkins_74,HAGEDORN1985,lissmann1965activity}. 
Ground structure: Our results show different (micro-) habitat preferences of different species of weakly electric fish.
During the day, pulsefish preferred plants and roots as shelter, but seldom between stones (fig.~\ref{fig:habitat_vs_flow}). In contrast, \textit{A. macrostomus} inhabited mostly stacked stones, but was also found in roots, rarely in plants. \textit{Eigenmannia} showed a preference for mainly roots, which corresponds with the literature \citep{Hopkins_74}. In pure sand habitats no fish were found. Since weakly electric fish are nocturnal animals \citep{lissmann1965activity} that seek shelter during the day \citep{Hopkins_74}, it seems likely that micro-habitats containing solely sand are poor hiding places for the three occurring species.
Habitat preferences and occurrence frequencies
Over-all, the habitat preferences of the individual species correspond with their occurrence frequencies. Since, \textit{Apteronotus} clearly preferred stony habitats, which was the most common habitat characteristic, \textit{Apteronotus} was by far the most frequent species. The preferred habitat of \textit{Eigenmannia} (roots), was less common. Consequently, compared with \textit{Apteronotus}, \textit{Eigenmannia} occurred less frequent. Pulsefish were most likely found in habitats containing plants and roots. Since plants were less frequent, because they mostly appeared on the shoreline, pulsefish were even less frequent. 


Water flow and water depth can also play a role in shelter or habitat selection \citep{aadland1993stream}. Pulsefish inhibited habitats with either very low water flow or no water flow at all (fig.~\ref{fig:habitat_vs_flow}).  Most of them were found in water depth between 20~and 40~cm (fig.~\ref{fig:habitat_vs_depth}). However, a clear preference for a certain water depth can not be seen, due to a small amount of recorded pulse-fish in diverse micro-habitats.
\textit{Apteronotus} also showed no preference, neither for water flow nor water depth. This species was found in a large flow range from stagnant to fast moving water and a variety of different water depths.
\textit{Eigenmannia} mainly was found in slow moving water, but individuals were also found in high flow areas. Regarding the water depth, \textit{Eigenmannia virescens} is the only species that seemed to prefer deeper water. However, this effect might be caused by measurement inaccuracies. Sometimes, especially if several animals were found within a micro-habitat, it was impossible to tell at which depth the individuals appeared. For each recording and micro-habitat, the denoted water depth corresponds only with the position of one individual (chpt.~\ref{sec:recordings}).

Over-all, \textit{Apteronotus} seems to be more opportunistic, occurring in a variety of micro habitats regardless of water depth and flow. Other species of weakly electric fish are rarely found in the preferred habitat of \textit{Apteronotus} (stones). This could indicate that next to same sex conspecifics \citep{raab2019}, \textit{Apteronotus} may also defend its shelter against other species. \textcolor{red}{Quelle}
To better understand the habitat preferences of the different species,  it is necessary to know why shelters are selected. Investigating natural foraging behavior and interspecific dominance of the different species, would be a good next step. Habitat preferences may depend on the food sources in these habitats. It is likely that the different species prefer different aquatic insect larvae \citep{marrero1991notas}. 

% --------------------------------------------------
% only apteronoti
% --------------------------------------------------

\subsection{Dominance and habitat preferences in \textit{Apteronotus}}

\textit{Apterontus macrostomus} is mainly found in stones or plants, which can also be observed under laboratory conditions \citep{raab2019,}, but we also found them in roots. There was a difference in the distribution of fee living males and females. While females showed no certain preference towards a stone or root habitat, males seem to prefer stony habitats. High EOD frequency is characteristic for dominance in male \textit{A. macrostomus}. The higher the frequency the more dominant the male \citep{raab2019}. We observed, that dominant male with an EOD frequency abobe 880 Hz exclusively inhabited stony habitats. This suggests that stony habitats are preferred hiding place over the day. Raab et al. showed in laboratory experiments, that dominant males defend their shelter against subordinated males. They also stay longer in their preferred day time habitat. But it is also observed that individuals switch to a new shelter in regular intervals over night as well as over the day \citep{raab2019}. It is not possible for us, to see if our recorded individuals are also switching shelters or habitats or they are more robust in the wild.  To better understand this, it is necessary to investigate the behavior in the wild over day and night to see which shelters are chosen.

% POSTER:
%
% 1. anzahl tiere, die zusammen auftreten (eigenmannia in gruppen von 3)
% 2. boxplots, die Verteilung zeigen statt nur mittlere anzahl
%    dann mit statistik !
% 3. bei apteronotus/dominanz mit histogram ?
%    statt t-test ROC analyse ?
% FOTOS, FISH Finder, EODdistr, WAVEFORM + CLUSTER, distrMaleFemale, FLow (adjusted)


% TO DO:
% ALLE:
% Apteronotus und Eigenmannia in allen abb. kursiv
%
% JULE:
% micro-habitat characteristica statt habitat count
% depth legende, selbe achsen, jeder plot einzeln
%
% MAX:
% stone, rootes (n=...)  unter jeden tick --> balkendiagram
% stones - stacked stones nicht gestackte balken + stones-NoStones und singleStones-steacked-Stones in paaren
% n = x ohne n = --> nur die zahlen
%
% JACQUI:
% zu jacquis plot auch histogram, water flow hist für jede art daneben
% signifikanzbalken
% sagen, dass bei männchen, weibchen nur die punkte aus jaquies plot sind
% + hist bei jacquies plot (männchen, weibchen)